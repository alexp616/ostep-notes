\chapter{Address Translation}

Hardware provides \textbf{address translation}, which quickly and efficiently translates virtual addresses into physical addresses.

This is only part of the full implementation the OS uses to abstract and manage memory.

\section{Assumptions}

For now, assume that:

\begin{enumerate}
    \item Process' address space maps continuously onto physical memory.
    \item Address space is strictly less than the size of physical memory.
    \item Every process' address space is the same size.
\end{enumerate}

\section{An Example \& Dynamic (Hardware-based) Relocation \& Hardware Support: A Summary}

Say a process has an address space that starts at 0 and ends at 16KB. However, in physical memory, it occupies the region 32KB - 48KB.

\textbf{Dynamic relocation}, or \textbf{base and bounds}, stores the start (base) and size (bound) of a process' physical address space somewhere (in registers). The process' virtual address space starts at 0, and translation is just $address_{physical} = address_{virtual} + base$. If the resulting address is between $base$ and $base + bound$, then it is a valid address.

The two registers storing base and bound are referred to as the \textbf{memory manage unit (MMU)}. We will be adding a lot more functionality to this MMU.

A single bit on the CPU indicates whether the CPU is in user mode or kernel mode. The base and bound registers can only be modified with privileged instructions.

The CPU generates \textbf{exceptions} when a process tries to access memory illegally. The user program is paused and the exception handled by the OS' \textbf{exception handler}, likely ending up in the process being killed.

\section{Operating System Issues}

The OS maintains a \textbf{free list} to find room for newly created processes, and to update free space whenever a process is ended.

The base and bounds registers, like other CPU registers, must be saved to the process' PCB when a context switch occurs.

And OS can move around a process' address space when the process is in the ready state if needed. This is done by just changing the base register in the process' PCB.

Exception handlers functions to be called when an exception occurs, and these are initialized at boot.

\section{Summary}

Dynamic relocation as we know it right now sucks - there is a lot of \textbf{fragmentation}, since a process might not actually use all of its address space, leading to wasted memory.