\chapter{Swapping: Mechanisms}

We want to support address spaces larger than what can fit in memory, so that a program doesn't have to worry about if there is enough room in memory to sustain it.

The answer to this is swap space, which swaps memory between RAM and disk, preferably the less frequently used memory.

\section{Swap Space}

The size of the swap space and RAM determines the maximum number of pages that a system can use, 

Swap space is not the only location a process may frequently swap pages between - at process start, it loads code from disk, and code pages can be freed to make space for other pages since they can be found on disk anyways.

\section{The Present Bit \& The Page Fault}

When a TLB miss occurs, the kernel may find that the corresponding PTE states that the page is not present in physical memory - its present bit is 0. This is called a \textbf{page} fault, generating a trap, blocking the process, and causing the OS to jump to the \textbf{page-fault} handler.

The PTE holds data to help find the disk address of a PFN. The page is fetched from disk, the OS updates the present bit and the PFN of the PTE, and retries the instruction. This process is called a \textbf{page in}. 

Page ins can cause a TLB miss, and that process is described earlier.

\section{What If Memory Is Full?}

If physical memory is full on a page in, then the OS needs to \textbf{page out} one or more pages to make room for the incoming page. How it decides which pages to page out is known as the \textbf{page-replacement policy}. (next chapter)

\section{Page Fault Control Flow}

This is almost all of the steps taken when a process tries to read memory:

\begin{enumerate}
    \item Process loads memory at virtual address $i$.
    \item VPN is computed using leading bits.
    \item Check if the TLB has the corresponding PFN. If it does, jump to step 9.
    \item If the TLB doesn't have the translation, then search the process' page table. This search is described in the chapter about page directories.
    \item If the PTE says that the page is present and valid, then jump to step 9 with the PFN.
    \item If the page is invalid, have the OS deal with the process.
    \item If the page is valid and not present, then load the frame from disk into a physical frame if there are free physical frames.
    \item If there aren't free physical frames, then find a physical frame to evict with a page-replacement policy. Swap the page on the disk and the evicted page. Jump to step 3.
    \item Compute the physical address, access the memory, and return it to the process.
\end{enumerate}

\section{When Replacements Really Occur}

There should always be a small amount of memory free, so most operating systems have two values: a \textbf{high watermark} (HW) and a \textbf{low watermark} (LW). When there are less than LW pages available, a background thread called the \textbf{swap daemon} or \textbf{page daemon} evicts pages to swap space until there are HW pages available.

This allows pages to be swapped in clusters, increasing efficiency of disk I/O.

\section{Summary}

