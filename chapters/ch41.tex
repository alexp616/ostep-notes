\chapter{Fast File System (FFS)}

When UNIX was first introduced, its file system involved just had a superblock, inodes, and datablocks.

\section{The Problem: Poor Performance \& FFS: Disk Awareness Is The Solution}

This file system performed poorly because:

\begin{itemize}
    \item The disk's storage was treated just like RAM.
    \item The disk would become fragmented, as seen in heap allocations. (This gave rise to disk \textbf{defragmentation} tools, which reorganize data to place files contiguously).
    \item The block size, 512 bytes, was too small. It helped reduce \textbf{internal fragmentation}, fragmentation within the block, but was terrible for bandwith.
\end{itemize}

Researchers built a file system called the \textbf{Fast File System}, which followed the idea to make data structures and allocation structures more ``disk aware''. They kept the same API calls, but changed the implementations.

\section{Organizing Structure: The Cylinder Group}

